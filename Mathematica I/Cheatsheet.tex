\documentclass[a4paper,12pt]{article}

%%% Работа с русским языком
\usepackage{cmap}					% поиск в PDF
\usepackage[T2A]{fontenc}			% кодировка
\usepackage[utf8]{inputenc}			% кодировка исходного текста
\usepackage[english,russian]{babel}	% локализация и переносы
%%
\usepackage{cmap} 
\usepackage{ gensymb }
\usepackage[unicode]{hyperref}
\usepackage{ textcomp }
\usepackage{datetime}
\usepackage{physics}
\usepackage{cancel}
\usepackage{mathtools}
\usepackage[margin=0.7in]{geometry}
\usepackage{fancyhdr}
\pagestyle{fancy}
%%

%%% Дополнительная работа с математикой
\usepackage{amsfonts,amssymb,amsthm,mathtools} % AMS
\usepackage{amsmath}
\usepackage{icomma} % "Умная" запятая: $0,2$ --- число, $0, 2$ --- перечисление

%% Теоремы

%%

\fancyhead[L]{\footnotesize Уравнения математической физики}
\fancyhead[RO]{Последняя компиляция: \today }
\fancyfoot[R]{\thepage}
\fancyfoot[C]{}




%% Номера формул
%\mathtoolsset{showonlyrefs=true} % Показывать номера только у тех формул, на которые есть \eqref{} в тексте.

%% Шрифты
\usepackage{euscript}	 % Шрифт Евклид
\usepackage{mathrsfs} % Красивый матшрифт

%% Свои команды
\DeclareMathOperator{\sgn}{\mathop{sgn}}

%% Перенос знаков в формулах (по Львовскому)
\newcommand*{\hm}[1]{#1\nobreak\discretionary{}
	{\hbox{$\mathsurround=0pt #1$}}{}}

%%% Работа с картинками
\usepackage{graphicx}  % Для вставки рисунков
\graphicspath{{images/}{images2/}}  % папки с картинками
\setlength\fboxsep{3pt} % Отступ рамки \fbox{} от рисунка
\setlength\fboxrule{1pt} % Толщина линий рамки \fbox{}
\usepackage{wrapfig} % Обтекание рисунков и таблиц текстом

%%% Работа с таблицами
\usepackage{array,tabularx,tabulary,booktabs} % Дополнительная работа с таблицами
\usepackage{longtable}  % Длинные таблицы
\usepackage{multirow} % Слияние строк в таблице


%%% Заголовок
\title{Теоретический минимимум , Математика I}
\date{\today}


\begin{document} % конец преамбулы, начало документа
	
	\maketitle
	\section{Интегрирование рациональных функций}
	\begin{equation}
	\int \frac{M x+N}{a x^{2}+b x+c} d x=\frac{M}{2 a} \ln \left(a x^{2}+b x+c\right)+\frac{2 a N-M b}{a \sqrt{4 a c-b^{2}}}  \operatorname{arctg} \frac{2 a x+b}{\sqrt{4 a c-b^{2}}}+C 
	\end{equation}
	
	\begin{equation}
	{\left(4 a c-b^{2}\right) u_{n}=\frac{2 n-3}{n-1} \cdot 2 a u_{n-1}+\frac{1}{n-1} \cdot \frac{2 a x+b}{\left(a x^{2}+b x+c\right)^{n-1}}} \quad  {u_{n}=\int \frac{d x}{\left(a x^{2}+b x+c\right)^{n}}}
	\end{equation}

	
	\begin{equation}
	\begin{aligned}
	J_{n}=\int \frac{d x}{\left(x^{2}+a^{2}\right)^{n}}, \quad n \in N \quad
	J_{n+1}=\frac{1}{2 n a^{2}}\left(\frac{x}{\left(x^{2}+a^{2}\right)^{n}}+(2 n-1) J_{n}\right)\\
	J_{1}=\int \frac{d x}{x^{2}+a^{2}}=\frac{1}{a} \operatorname{arctg} \frac{x}{a}+C
	\end{aligned}
	\end{equation}
	
	

	
	В интегралах вида $x^{m}\left(a x^{n}+b\right)^{-p}$ полезна замена $x^{\sigma}=t$ где $ \sigma=\operatorname{QCD}\left(m+1, n\right) $
	
	\par
	
	Ecли в знаменатели разлагаются на простые множители первой степени, при разложении дроби на простейшие удобна формула:
	
	$$
	\frac{\varphi(x)}{\psi(x)}=\sum_{k=1}^{n} \frac{\varphi\left(a_{k}\right)}{\psi^{\prime}\left(a_{k}\right)} \frac{1}{x-a_{k}}
	$$
	Где $a_{1}, a_{2} \dots \ldots a_{n} -$ корни полинома $ \psi(x) $.
	
\end{document} % конец документа
